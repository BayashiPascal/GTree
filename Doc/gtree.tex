\section*{Introduction}

GTree is a C library providing structures and functions to manipulate tree structures.\\ 

A GTree is a structure containing a pointer toward its parent, a void* pointer toward user's data and a GSet of subtrees. The GTree offers the same interface has a GSet to manipulate its subtrees. It also provides a function to cut the GTree from its parent.\\

The library provides also three iterators to run through the trees: GTreeIterDepth, GTreeIterBreadth, GTreeIterValue which step, respectively, in depth first order, breadth first order and value (sorting value of the GSet of subtrees) first order.\\

It uses the \begin{ttfamily}PBErr\end{ttfamily} and \begin{ttfamily}GSet\end{ttfamily} libraries.\\

\section{Interface}

\begin{scriptsize}
\begin{ttfamily}
\verbatiminput{/home/bayashi/Coding/GTree/gtree.h}
\end{ttfamily}
\end{scriptsize}

\section{Code}

\subsection{pbmath.c}

\begin{scriptsize}
\begin{ttfamily}
\verbatiminput{/home/bayashi/Coding/GTree/gtree.c}
\end{ttfamily}
\end{scriptsize}

\subsection{pbmath-inline.c}

\begin{scriptsize}
\begin{ttfamily}
\verbatiminput{/home/bayashi/Coding/GTree/gtree-inline.c}
\end{ttfamily}
\end{scriptsize}

\section{Makefile}

\begin{scriptsize}
\begin{ttfamily}
\verbatiminput{/home/bayashi/Coding/GTree/Makefile}
\end{ttfamily}
\end{scriptsize}

\section{Unit tests}

\begin{scriptsize}
\begin{ttfamily}
\verbatiminput{/home/bayashi/Coding/GTree/main.c}
\end{ttfamily}
\end{scriptsize}

\section{Unit tests output}

\begin{scriptsize}
\begin{ttfamily}
\verbatiminput{/home/bayashi/Coding/GTree/unitTestRef.txt}
\end{ttfamily}
\end{scriptsize}


